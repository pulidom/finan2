\documentclass[10pt,hyperref={colorlinks = true,linkcolor = blue}]{beamer}
%\documentclass[10pt]{beamer}
% Setup appearance:

\usetheme{default}
\usefonttheme[onlylarge]{structurebold}
%\setbeamerfont*{frametitle}{size=\normalsize,series=\bfseries}
\setbeamerfont*{frametitle}{size=\large,series=\bfseries}
\setbeamertemplate{navigation symbols}{}
%\setbeamersize{text margin left=0.5cm, text margin right=0.5cm} 
\setbeamersize{text margin left=0.5cm, text margin right=0.5cm} 

\usefonttheme[onlymath]{serif}

\setbeamertemplate{blocks}[rounded][shadow=true]
\newtheorem{formula}{}
\setbeamercolor{block body formula}{bg=green}

\usepackage{xcolor}
%\usepackage[colorlinks]{hyperref}
%\hypersetup{
%  colorlinks = true,
%  urlcolor = {blue},
%    linkbordercolor = {blue}
%    linkbordercolor = {blue}
%}


% verbatim con small font
\usepackage{etoolbox}
\makeatletter
\patchcmd{\@verbatim}
  {\verbatim@font}
  {\verbatim@font\small}
  {}{}
\makeatother
\usepackage{pythonhighlighting}

\usepackage[mathscr]{eucal}

%\def\srci{\begin{mdframed}[hidealllines=true,backgroundcolor=gray!25]\begin{verbatim}}
%\def\srcf{\end{verbatim}\end{mdframed}}
\usepackage{verbatim}
\usepackage{verbatimbox}
\usepackage{listings}
\usepackage{color}

% Codigo-------------
%https://stackoverflow.com/questions/3175105/inserting-code-in-this-latex-document-with-indentation
\usepackage{listings,newtxtt}
\lstset{basicstyle=\ttfamily, keywordstyle=\bfseries}
\definecolor{dkgreen}{rgb}{0,0.6,0}
\definecolor{gray}{rgb}{0.5,0.5,0.5}
\definecolor{mauve}{rgb}{0.58,0,0.82}

\lstset{frame=tb,
  language=Python,
  aboveskip=3mm,
  belowskip=3mm,
  showstringspaces=false,
  columns=flexible,
  basicstyle={\footnotesize\ttfamily},
  numbers=none,
  numberstyle=\tiny\color{gray},
  keywordstyle=\color{blue},
  commentstyle=\color{dkgreen},
  stringstyle=\color{mauve},
  breaklines=true,
  breakatwhitespace=true,
  tabsize=3
}
%\begin{lstlisting}
%// Hello.java
%import javax.swing.JApplet;
%import java.awt.Graphics;
%
%public class Hello extends JApplet {
%    public void paintComponent(Graphics g) {
%        g.drawString("Hello, world!", 65, 95);
%    }    
%}
%\end{lstlisting}
%---codigo----------

\newenvironment{wideframe}
{
 % not too sure, but may be needed if you have a background image
 % that should not appear on this kind of frame:
  \setbeamertemplate{navigation symbols}{}
%  \setbeamerfont*{frametitle}{size=\large,series=\bfseries}

 % locally set margins to zero: (notice the use of \bgroup ... \egroup
 % to limit the scope of the geometry restriction
 % where curly brackets {} aren't possible)
%\setbeamersize{text margin left=0cm, text margin right=0cm} 
 \bgroup \newgeometry{left=0cm,right=0cm,top=0.5cm} 
 \begin{frame}[plain]
}
{
 \end{frame}
 \egroup
}
%\usepackage{empheq} %box equations?
%\usepackage[many]{tcolorbox}

% Standard packages
\newcommand*{\boxedcolor}{red}
\makeatletter
\renewcommand{\boxed}[1]{\textcolor{\boxedcolor}{%
    \fbox{\normalcolor\m@th$\displaystyle#1$}}}
\makeatother
    
\usepackage[english]{babel}
\usepackage[latin1]{inputenc}
\usepackage{times}
\usepackage[T1]{fontenc}

\usepackage{tikz}
\usetikzlibrary{calc,tikzmark,positioning}
%\newcommand\tikzmark[1]{\tikz[overlay,remember picture] \node (#1) {};}
\graphicspath{{./fig/}}
% Setup TikZ

\newcommand\Slash{\char`\\}

% comandos para hacer boxes
\usepackage{xcolor}
%\usepackage{lipsum}
%\newcommand*{\mybox}[1]{\colorbox{gray!30}{\parbox{.98\linewidth}{#1}}}
\newcommand{\mybox}[1]{\colorbox{gray!30}{\parbox{.98\linewidth}{#1}}}
\usepackage[framemethod=tikz]{mdframed}
\def\bxi{\begin{mdframed}[hidealllines=true,backgroundcolor=blue!20]}
\def\bxf{\end{mdframed}}

\newcommand{\cter}[1]{\mybox{\$\,#1}}
\newcommand{\ctercorto}[1]{\colorbox{gray!30}{\$\,#1}}
%\newcommand{\ctermath}[1]{\colorbox{gray!30}{\parbox{.98\linewidth}{$\displaystyle #1$}}}
\newcommand{\cpyt}[1]{\mybox{$>>>$\,#1}}

\usepackage{tikz}
\usetikzlibrary{arrows}
\tikzstyle{block}=[draw opacity=0.7,line width=1.4cm]
\renewcommand{\baselinestretch}{1.2}


%\usepackage{color}
%\usepackage{arrow}
%\def\arrowscale{0.011}
% pseudo code
\usepackage{algpseudocode,algorithm,algorithmicx}
\newcommand*\Let[2]{\State #1 $\gets$ #2}
\renewcommand{\algorithmicrequire}{\textbf{Input:}}
\renewcommand{\algorithmicensure}{\textbf{Output:}}

\def\rp{\textcolor{red}} %corrections, red pencil
\def\bp{\textcolor{blue}}
%algunas funciones definidas
%\def\mi{\begin{equation}}
%\def\mf{\end{equation}}
\def\mi{$$}
\def\mf{$$}
\def\med{\overline}
%\def\mean{\overline}
\def\reff#1{(\ref{#1})}
\def\mean{\mathcal{E}}
\def\mbf{\mathbf}\def\mrm{\mathrm}
\def\gwdunit{\, \mrm{m \, s^{-1} \, day^{-1}}}
\def\vunit{\, \mrm{m \, s^{-1}}}
\def\punit{\, \mrm{hPa}}
\newcommand{\ud}{\mathrm{d}}
\newcommand{\dg}{$^\circ$ } % en textos 

% algunas funciones x
\renewcommand{\v}[1]{\ensuremath{\mathbf{#1}}} % for vectors
\newcommand{\gv}[1]{\ensuremath{\mbox{\boldmath$ #1 $}}}  % for vectors of Greek letters
\newcommand{\uv}[1]{\ensuremath{\mathbf{\hat{#1}}}} % for unit vector
\newcommand{\abs}[1]{\left| #1 \right|} % for absolute value
\newcommand{\avg}[1]{\left< #1 \right>} % for average
\let\underdot=\d % rename builtin command \d{} to \underdot{}
\renewcommand{\d}[2]{\frac{\mrm{d} #1}{\mrm{d} #2}} % for derivatives
\newcommand{\dd}[2]{\frac{\mrm{d}^2 #1}{\mrm{d} #2^2}} % for double derivatives
\newcommand{\pd}[2]{\frac{\partial #1}{\partial #2}} % for partial derivatives
\newcommand{\pdd}[2]{\frac{\partial^2 #1}{\partial #2^2}}  % for double partial derivatives
\newcommand{\pdc}[3]{\left( \frac{\partial #1}{\partial #2}
 \right)_{#3}} % for thermodynamic partial derivatives
\newcommand{\ket}[1]{\left| #1 \right>} % for Dirac bras
\newcommand{\bra}[1]{\left< #1 \right|} % for Dirac kets
\newcommand{\braket}[2]{\left< #1 \vphantom{#2} \right|
 \left. #2 \vphantom{#1} \right>} % for Dirac brackets
\newcommand{\matrixel}[3]{\left< #1 \vphantom{#2#3} \right|
 #2 \left| #3 \vphantom{#1#2} \right>} % for Dirac matrix elements
\newcommand{\grad}[1]{\gv{\nabla} #1} % for gradient
\let\divsymb=\div % rename builtin command \div to \divsymb
\renewcommand{\div}[1]{\gv{\nabla} \cdot #1} % for divergence
\newcommand{\curl}[1]{\gv{\nabla} \times #1} % for curl
\let\baraccent=\= % rename builtin command \= to \baraccent
\renewcommand{\=}[1]{\stackrel{#1}{=}} % for putting numbers above =

\def\delequal{\overset{\scriptstyle\Delta}{=}}
%\def\esp\mathcal{E}
\newcommand{\expec}[1]{\mathcal{E}\left( #1 \right)}
\newcommand{\esp}[1]{\mathcal{E}\left(\left. #1 \right|\v y_{1:k}\right)} % esperanza
\newcommand{\mdl}[1]{\mathcal{M}\left( #1 \right)} % modelo
\renewcommand{\H}[1]{\mathcal{H}\left( #1 \right)} %observational oper
\newcommand{\transp}{\top}
\newcommand{\rkhs}{\mathcal{F}}

\newcommand{\dynm}{\mathcal{M}}
\newcommand{\obsm}{\mathcal{H}}
\newcommand{\gaus}{\mathcal{N}}
\newcommand{\normal}{\mathcal{N}}


\def\bs{\boldsymbol}
\def\ci{\begin{comment}}
\def\cf{\end{comment}}
\def\mbf{\mathbf}
\def\mrm{\mathrm}

% Author, Title, etc.
\unitlength 1cm

\title[Monge barycenter problem] 
{%
  \huge Programaci\'on  \\[3ex]
\Large
  Monge barycenter problem}
  %\normalsize
  %combining variational data assimilation  with  particle filters
%}

\author[Pulido]
{Dr.  Manuel Pulido }

\institute
{ 

  \small  $^1$ FaCENA, Universidad Nacional del Nordeste, Argentina\\
  \small  $^2$ CONICET, Argentina\\
  \small  $^3$ Data Assimilation Research Centre, University of Reading, UK\\
  \rp{Oficina: 1er Piso. FaCENA. Campus Libertad.}\\
  \rp{pulido@exa.unne.edu.ar}\\
    \rp{https://pulidom.github.io/}

}
\date{}

% The main document

\begin{document}


\begin{frame}{Motivacion}
%\vskip -1cm

  Supongamos un conjunto de variables de entrada, $z$, y una salida $x$.

  La idea seria ver como depende la variable de salida $x$ con las variables de entrada $z$.

  En general este problema consiste en inferir la densidad condicional $p(x|z)$.

A partir de los datos que son un conjunto de $n$ pares, $(x_i,z_i)$, los objetivos son:

\begin{itemize}
\item Poder evaluar analiticamente (parametricamente?) a $p(x|z)$.
\item Dado un valor del target $z^*$, generar un conjunto de muestras ${x_j}$ a partir de $p(x|z^*)$
\end{itemize}

\end{frame}


% David Pine. Ajuste de curvas
% Landau and paez
% kiusalass_numericalMethodsWithPython.pdf
% Fangohr. Raices.
%%%%%%%%%%% FILMINA 1 %%%%%%%%%%%%%%


%%%%%%%%%%% FILMINA 1 %%%%%%%%%%%%%%
\begin{frame}[fragile]{Problema de transporte de Monge del baricentro}
Buscamos un mapa que elimine de la densidad de $x$ toda la variabilidad explicable por $z$. Esto es lo que hacemos cuando obtenemos $p(x|z)$ dado un $z^*$.

Para encontrar ese mapa pongo dos condiciones

\[\min_{y=T(x,z)} \mathbb E_\pi [ c(x,y) ] \]

sujeto a que $y$ sea independiente de $z$.

En algun sentido decimos a que no quede mas informacion de $x$ en $z$.


  \end{frame}

%%%%%%%%%%% FILMINA 1 %%%%%%%%%%%%%%
 \begin{frame}[fragile]{Densidad condicional: Metodo generativo con OTBP}

   Una vez que encontre el $T$ lo puedo usar para generar muestras de la condicional.
   

   Dado los datos $\{x_i,z_i\}_{1:n}$ puedo transformar las muestras  $y_i=T(x_i,z_i)$.

   Si tengo un $z^*$ sobre el cual quiero generar muestras de $x$ condicionadas a $z^*$ 

 Antitransformo sobre x a $x^*_i=T^{-1}(y_i,z^*)$ esto me dar $N$ muestras posibles de la variabilidad de $x$ dado $z^*$.

 \end{frame}

 \begin{frame}[fragile]{Problema primal relajado}

   Funci�n Objetivo
   
\[\min_{\mu, \pi} \int c(x, y) \pi(x, y) \, dx \, dy \]

Restricciones

\[\int \pi(x, y) \, dy = \rho(x), \quad \int \pi(x, y) \, dx = \mu(y) \]

donde $\mu(y)$ es la distribucion del baricentro y la $\pi$ es la distribucion conjunta que acopla $\rho$ con $\mu$

 \end{frame}

 \begin{frame}[fragile]{Problema de adversarios empirico}
Dadas las muestras $(x_i, z_i)_{i=1}^N$, proponemos trabajar en un espacios de fucniones $\mathcal F$ para $z$ y  $\mathcal G$ para $y$
   
   
\[\min_{y_i = T(x_i, z_i)} \max_{g, f}  \left\{\frac{1}{n} \sum_{i=1}^{n} \left[ c(x_i, y_i) + \lambda \, g(y_i) f(z_i) \right] \right\}, \]

Restricciones: $\sum_{i=1}^{n} f(z_i) = 0, \quad \|f\| = \|g\| = 1 $

Primeramente necesitamos encontrar las $g$ y $f$ en el espacio de funciones que maximizan la funcion de costo. 

La independencia impone una condicion de ``sin correlaci\'on'' entre la $g$ y la $f$. Si $y$ y $z$ son independientes se tiene que

\[\mathbb E_\pi [g(y) f(z)] = 0 \]
 \end{frame}
\begin{frame}[fragile]{Test functions}
  Vamos a restringir la busqueda  de las funciones \(f\) y \(g\)  a  espacios de dimension finita (\(\mathcal{F}\) y \(\mathcal{G}\)).

 Solo consideramos combinaciones lineales de un conjunto finito de funciones base ("features" o caracter�sticas).

\[
f(z) = F(z)a = \sum_{j=1}^{m_z} F^j(z) \, a_j, \quad \quad g(y) = G(y)b = \sum_{k=1}^{m_y} G^k(y) \, b_k
\]
donde:
   \(F(z)\) y \(G(y)\) son vectores de funciones base predefinidas (por ejemplo, polinomios, funciones trigonom�tricas, kernels, etc.).
   \(a \in \mathbb{R}^{m_z}\) y \(b \in \mathbb{R}^{m_y}\) son los vectores de coeficientes a determinar.
   \(m_z\) y \(m_y\) son las dimensiones de estos espacios de caracter�sticas.

 \end{frame}
\begin{frame}[fragile]{Transformo al espacio ortogoanal}

   Se reemplaza $G(y)$ por un operador ortogonal  $Q_y(y)$ que genere el mismo rango efectivo a traves de SVD,


\[
\mathbf{G}_0 = \mathbf{U} \boldsymbol{\Sigma} \mathbf{V}^T
\]

donde $\v G_0$ es la matriz evaluada en los puntos $y_{0j}$.

La cantidad de dimensiones de $Q_y(y)$ es tal que genere gran parte de la variabilidad con $m_y$ tal que $\sum_k^{m_y} \sigma_k^2 > 0.99$


Se define una nueva matriz de transformaci�n \( \mathbf{B}^y \) de tama�o \( m_y \times n_y \) que proyecta los coeficientes originales al espacio reducido y ortogonal:

\[
\mathbf{B}^y_{jk} = \frac{1}{\sigma_k} \mathbf{V}_{jk}, \quad \text{para } j=1,\dots,m_y; \ k=1,\dots,n_y
\]

 \end{frame}
\begin{frame}[fragile]{Problema de adversarios en el  espacio ortogonal}

El nuevo operador de caracter�sticas ortogonales \( Q_y \) se define como:

\[
Q_y(y) = G(y) \, \mathbf{B}^y
\]
Entonces hemos representado al problema de maximizacion de funciones en un espacio finito y a traves de coeficientes que tienen que ser determinados a traves de la maximizacion.

Entonces ahora el problema min-max lo tenemos en $a,b,y_l$,

\[\min_{\{y_i\}} \left\{ \max_{a,b} \sum_{i=1}^{n} c(x_i, y_i) + \lambda \sum_{h=1}^{n_z} \sum_{l=1}^{n_y} \left( \sum_{i=1}^{n} Q_z^{h}(z_i) Q_y^{l}(y_i) \right) a_h b_l, \quad \|a\| = \|b\| = 1 \right\}\]

\end{frame}
\begin{frame}[fragile]{Determinacion explicita del $a$ y $b$ que maximizan}
Se puede determinar en forma explicita la maximizacion en esencia $a$ y $b$ se deben alinear con las primeras componentes principales a izquierda y derecha.

Nuevamente entonces hacemos SVD sobre la matriz $\v a^\top \v A \v b = \v a^\top \v Q_z^\top \v Q_y \v b = \v U \gv \Sigma \v V^\top$.

Entonces las funciones son: $ f(z)  = \v Q_z(z) a; \quad g(x) = \v Q_y(y) \v b$. 


\end{frame}
\begin{frame}[fragile]{Optimizacion con flujos de part\'iculas}
Dados $a$ y $b$ terminamos en un problema de minimizacion dado por

\[\min_{\{y_i\}}  \sum_{i=1}^{n} c(x_i, y_i) + \lambda  \, \|A(y)\| \]

Problema de optimizacion con mapas que mueven {\bf todas} las particulas
\[y_i^{n+1} = y_i^n - \eta^n \left[ \frac{1}{n} \nabla_y c(x_i, y)\big|_{y_i^n} + \lambda \, a^{\prime} \, \nabla_y A\big|_{y_i^n} b \right] \]

esto lo puedo ver como un descenso de gradientes.



\[\nabla_y A^{hl}\Big|_{y_i^n} = Q_z^h(z_i) \sum_j \nabla G^j(y)\big|_{y_i} B_{jl}^y\]

\end{frame}

\end{document}
