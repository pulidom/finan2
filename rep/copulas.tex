\documentclass[12pt, a4paper]{myarticle}
\usepackage[spanish]{babel}
\usepackage[utf8]{inputenc}
\usepackage{amsmath, amssymb, amsthm}
\usepackage{graphicx}
\usepackage{xcolor}
\usepackage{hyperref}

\title{Aplicación de Cópulas al Arbitraje Estadístico con Pares}
\author{Análisis Cuantitativo}
\date{}

\begin{document}

\maketitle

\section{Introducción}
El arbitraje estadístico con pares (\textit{pairs trading}) tradicional se basa en modelos de cointegración o regresión. Las cópulas ofrecen un enfoque más sofisticado al modelar la \textbf{estructura de dependencia completa} entre dos activos, capturando relaciones no lineales y dependencias de cola que los métodos tradicionales pasan por alto.

\section{Marco Teórico}

\subsection{Definición de Cópula}
Una cópula \( C \) es una función de distribución multivariada con marginales uniformes en el intervalo \([0,1]\). Según el \textbf{Teorema de Sklar}, para dos variables aleatorias \( X \) y \( Y \) con funciones de distribución conjunta \( F_{XY} \) y marginales \( F_X \) y \( F_Y \), existe una cópula \( C \) tal que:
\[
F_{XY}(x,y) = C(F_X(x), F_Y(y))
\]
La densidad de la cópula se define como:
\[
c(u,v) = \frac{\partial^2 C(u,v)}{\partial u \partial v}
\]
donde \( u = F_X(x) \) y \( v = F_Y(y) \).

\section{Proceso de Implementación}

\subsection{Paso 1: Preprocesamiento de Datos}

\subsubsection{Transformación de Precios}
Para un par de activos \( A \) y \( B \), calculamos los rendimientos logarítmicos:
\[
r_{A,t} = \ln(P_{A,t}) - \ln(P_{A,t-1})
\]
\[
r_{B,t} = \ln(P_{B,t}) - \ln(P_{B,t-1})
\]

\section{Transformación a Uniformes}

\subsection{Marginales Empíricas}

Para cada activo, estimamos la distribución acumulada marginal de rendimientos:

\[
\hat{F}_A(r) = \frac{1}{T} \sum_{t=1}^T \mathbb{I}_{\{r_{A,t} \leq r\}}
\]

\textbf{Ejemplo:} Si tenemos 100 observaciones y 30 de ellas son \(\leq 0.02\), entonces:
\[
\hat{F}_A(0.02) = \frac{30}{100} = 0.30
\]
\subsection{Transformación a Uniformes}

Aplicamos la transformada integral de probabilidad:

\[
u_t = \hat{F}_A(r_{A,t}), \quad v_t = \hat{F}_B(r_{B,t})
\]

Donde:
\begin{itemize}
    \item \( r_{A,t}, r_{B,t} \): \textbf{Rendimientos observados} (datos originales)
    \item \( \hat{F}_A, \hat{F}_B \): \textbf{Funciones de distribución acumulada} empíricas o paramétricas
    \item \( u_t, v_t \): \textbf{Variables uniformes} en [0,1] resultantes de la transformación
\end{itemize}
\subsection{Paso 2: Modelado con Cópulas}

\subsubsection{Familias de Cópulas Common}

\begin{itemize}
    \item \textbf{Gaussiana}: 
    \[
    C_{\rho}(u,v) = \Phi_{\rho}(\Phi^{-1}(u), \Phi^{-1}(v))
    \]
    
    \item \textbf{\textit{t} de Student}:
    \[
    C_{\rho,\nu}(u,v) = t_{\rho,\nu}(t_{\nu}^{-1}(u), t_{\nu}^{-1}(v))
    \]
    
    \item \textbf{Clayton}:
    \[
    C_{\theta}(u,v) = \left( \max\{u^{-\theta} + v^{-\theta} - 1, 0\} \right)^{-1/\theta}
    \]
    
    \item \textbf{Gumbel}:
    \[
    C_{\theta}(u,v) = \exp\left( -\left( (-\ln u)^{\theta} + (-\ln v)^{\theta} \right)^{1/\theta} \right)
    \]
\end{itemize}

\subsubsection{Estimación de Parámetros}
Los parámetros se estiman mediante máxima verosimilitud:
\[
\hat{\theta} = \arg\max_{\theta} \sum_{t=1}^{T} \ln c(u_t, v_t; \theta)
\]

\subsection{Paso 3: Señales de Trading}

\subsubsection{Probabilidades Condicionales}
Definimos las señales usando derivadas de la cópula:

\begin{align*}
\text{Probabilidad condicional para venta:} & \quad P(V > v | U = u) = 1 - \frac{\partial C(u,v)}{\partial u} \\
\text{Probabilidad condicional para compra:} & \quad P(V \leq v | U = u) = \frac{\partial C(u,v)}{\partial u}
\end{align*}

\subsubsection{Reglas de Trading}

\begin{itemize}
    \item \textbf{Señal VENTA} (Vender A/Comprar B):
    \[
    P(V > v | U = u) < \tau_{\text{low}}
    \]
    \item \textbf{Señal COMPRA} (Comprar A/Vender B):
    \[
    P(V \leq v | U = u) < \tau_{\text{low}}
    \]
\end{itemize}
donde \( \tau_{\text{low}} \) es un umbral típicamente entre 0.05 y 0.10.

\subsection{Paso 4: Gestión de Riesgos}

\subsubsection{Salida por Take-Profit}
\[
P(V > v | U = u) \approx 0.5 \quad \text{o} \quad P(V \leq v | U = u) \approx 0.5
\]

\subsubsection{Salida por Stop-Loss}
\begin{itemize}
    \item \textbf{Basado en probabilidad}: La probabilidad condicional se vuelve más extrema
    \item \textbf{Basado en precio}: Pérdida máxima definida sobre el spread
\end{itemize}

\section{Ejemplo Numérico}

\subsection{Escenario}
\begin{itemize}
    \item Par: Acción A vs. Acción B
    \item Cópula: \textit{t}-Student con \( \rho = 0.8 \), \( \nu = 5 \)
    \item Umbral: \( \tau_{\text{low}} = 0.05 \)
\end{itemize}

\subsection{Señal de Trading}
\begin{align*}
u_t &= 0.98 \quad (\text{A tiene rendimiento muy alto}) \\
v_t &= 0.60 \quad (\text{B tiene rendimiento moderado}) \\
P(V > 0.60 | U = 0.98) &= 0.03 < 0.05
\end{align*}
\textbf{Decisión}: Vender A/Comprar B

\section{Consideraciones Prácticas}

\subsection{Ventajas}
\begin{itemize}
    \item Captura dependencias no lineales y de cola
    \item interpretación probabilística sólida de las señales
    \item Mejor gestión del riesgo en eventos extremos
\end{itemize}

\subsection{Desafíos}
\begin{itemize}
    \item Complejidad en la estimación y selección del modelo
    \item Requiere suficiente data histórica
    \item Mayor costo computacional
\end{itemize}

\end{document}
